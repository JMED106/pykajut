% File_name: T1_c1.1_q01
% Title: Pregunta 1
This is just an example:
\begin{equation}
e^{-i\pi} + 1 = 0.
\end{equation}
\begin{enumerate}
\Myitem Yeah! %enditem
\Myitem Nop. %enditem
\Myitem Not always. %enditem
\Myitem I've seen better ones. % Correct %enditem
\end{enumerate}

% File_name: T1_c1.1_q02
% Title: Pregunta 1
Enunciat de prova. Posem accents aquí.
\begin{enumerate}
\Myitem No funcionarà. % Correct %enditem
\Myitem Mmm... %enditem
\Myitem Ok. %enditem
\Myitem Doncs bé. %enditem
\end{enumerate}

% File_name: T1_c1.1_q03
% Title: Pregunta 1
Enunciado de prueba. Esta vez en castellano (español), no creo que
vaya la ñ.
\begin{enumerate}
\Myitem $\alpha$. % Correct %enditem
\Myitem $\beta$. %enditem
\Myitem $\gamma$. %enditem
\Myitem $\delta$. %enditem
\end{enumerate}

% File_name: T2_c2.1_q01
% Title: Pregunta 1
\graphicspath{{/home/jm/Escritorio/prueba/figs/}}
Enunciado correspondiente al bloque T2, sección 2.1, pregunta
1. Prueba de figura.
\begin{center}
\includegraphics[width=0.5\textwidth]{image0.png}
\end{center}
\begin{enumerate}
\Myitem Respuesta 1.% Correct %enditem
\Myitem Respuesta 2. %enditem
\Myitem Respuesta 3. %enditem
\Myitem Respuesta 4. %enditem
\end{enumerate}

% File_name: T2_c2.1_q02
% Title: Pregunta 1
Enunciado de la pregunta.
\begin{enumerate}
\Myitem Respuesta 1.% Correct %enditem
\Myitem Respuesta 2. %enditem
\Myitem Respuesta 3. %enditem
\Myitem Respuesta 4. %enditem
\end{enumerate}
